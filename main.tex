\documentclass[a4paper, 12pt]{article}
\usepackage[utf8]{inputenc}
\usepackage{mathtools}
\usepackage{amssymb}
\usepackage{enumitem}
\usepackage{parskip}
\usepackage{xfrac}
\usepackage{xcolor}
\usepackage{booktabs}
\usepackage{graphicx}
\usepackage[export]{adjustbox}

\graphicspath{ {images/} }

\newcommand{\tr}{^{\mathsf{T}}}
\newcommand{\N}{\mathbb{N}}
\newcommand{\R}{\mathbb{R}}
\newcommand{\Z}{\mathbb{Z}}
\newcommand{\Q}{\mathbb{Q}}
\newcommand{\half}{\sfrac{1\!}2}
\DeclarePairedDelimiter\abs{\lvert}{\rvert}
\DeclarePairedDelimiter\floor{\lfloor}{\rfloor}
\DeclareMathOperator{\GL}{GL}
\DeclareMathOperator{\interior}{int}
\DeclareMathOperator{\closure}{cl}
\DeclareMathOperator{\aut}{Aut}
\DeclareMathOperator{\rank}{rank}

\setlist[enumerate, 1]{leftmargin=0pt, label=\textbf{\arabic*.}}

\begin{document}

\begin{enumerate}

\item \begin{enumerate}

\item \(Q_3\) has 4 eigenvalues listed below, where the entries of the eigenvectors are 1 if the corresponding piece of the cube is red and \(-1\) otherwise:
\begin{center}
\begin{minipage}{11cm}
\(\lambda_1=3\) spanned by \(\left\{\,\includegraphics[valign=c, width=2cm]{1}\,\right\}\)

\(\lambda_2=1\) spanned by \(\left\{\,\includegraphics[valign=c, width=2cm]{2}, \includegraphics[valign=c, width=2cm]{3}, \includegraphics[valign=c, width=2cm]{4}\,\right\}\)

\(\lambda_3=-1\) spanned by \(\left\{\,\includegraphics[valign=c, width=2cm]{5}, \includegraphics[valign=c, width=2cm]{6}, \includegraphics[valign=c, width=2cm]{7}\,\right\}\)

\(\lambda_4=-3\) spanned by \(\left\{\,\includegraphics[valign=c, width=2cm]{8}\,\right\}\)
\end{minipage}
\end{center}

\item The Peterson graph is strongly regular with \(k=3\), \(\lambda=0\) and \(\mu=1\). By the proof of the rationality condition, 3 is an eigenvalue spanned by the all-ones vector, and the other two eigenvalues are 1 with multiplicity 5 and \(-2\) with multiplicity 4, with eigenspaces orthogonal to the all-ones vector.

The five rotations of
\begin{center}
\includegraphics[valign=c, width=3cm]{pet1} \hspace{2em}\(\begin{array}{lcr}\text{red}&\to&1\\\text{blue}&\to&-1\\\text{white}&\to&0\end{array}\)
\end{center}
are linearly independent and so span the eigenspace of the eigenvalue 1

The five rotations of
\begin{center}
\includegraphics[valign=c, width=3cm]{pet2} \hspace{2em}\(\begin{array}{lcr}\text{red}&\to&3\\\text{blue}&\to&-2\end{array}\)
\end{center}
span a four-dimensional space since
\[\rank\begin{pmatrix}
-2&3&-2&-2&3&-2&-2&3&3&-2\\
3&-2&3&-2&-2&-2&-2&-2&3&3\\
-2&3&-2&3&-2&3&-2&-2&-2&3\\
-2&-2&3&-2&3&3&3&-2&-2&-2\\
3&-2&-2&3&-2&-2&3&3&-2&-2\\
\end{pmatrix}=4\]
and hence span the eigenspace of the eigenvalue \(-2\).

\item \(K_n\) is \((n-1)\)-regular so it has an eigenvalue \(n-1\) spanned by the all-ones vector. Also, \(K_n+I=J\) so \(K_n\) has an eigenvalue \(-1\) with \((n-1)\)-dimensional eigenspace the vectors whose entries sum to 0.

\item Let \(\omega\) be an \(n\)th root of unity. Then \(v\coloneqq(1,\omega,\omega^2\dots,\omega^{n-1})\) is an eigenvector of \(C_n\) with eigenvalue \(\omega+\omega^{-1}\) since the sum of the entries of the neighbours of a vertex with entry \(\omega^i\) is \(\omega^{i+1}+\omega^{i-1}=(\omega+\omega^{-1})\omega^i\). Note \(\omega^{-1}=\overline\omega\) so the eigenvalue \(\omega+\omega^{-1}\) is real.

If \(\omega\notin\{1,-1\}\) then the eigenvectors \(v\) and \(u\coloneqq(1,\omega^{n-1},\omega^{n-2},\dots,\omega)\) are linearly independent, so \(\omega+\omega^{-1}\) has multiplicity 2.

If \(\omega\) is 1 or \(-1\) the two corresponding eigenvectors coincide, so the eigenvector 2 is spanned by \((1,1,\dots,1)\) and the (possible) eigenvector \(-2\) is spanned by \((1,-1,1,-1,\dots,1,-1)\).

\end{enumerate}

\item The theoretical and actual independence numbers for the four graphs above are:
\begin{center}
\begin{tabular}{@{}ccc@{}}\toprule
Graph&Theorem 2.12 bound&Actual\\\midrule
\(Q_3\)&4&4\\
Peterson&4&4\\
\(K_n\)&1&1\\
\(C_n\)&\(\displaystyle \frac{n\cos\left(\frac{\floor*{\frac n2}}n2\pi i\right)}{\cos\left(\frac{\floor*{\frac n2}}n2\pi i\right)-1}\)&\(\displaystyle\floor*{\frac n2}\)\\
\bottomrule
\end{tabular}
\end{center}
When \(n\) is even the bound for \(C_n\) is met.

\item Let \(G\) be a graph such that every two distinct vertices have exactly two common neighbours.

\begin{enumerate}

\item Let \(u\) and \(v\) be distinct adjacent vertices of \(G\). For each neighbour \(u'\) of \(u\) there are exactly two common neighbours of \(u'\) and \(v\) --- one is \(u\) and the other we label as \(v'\). This induces a mapping \(\phi\colon u'\mapsto v'\) between the neighbourhoods of \(u\) and \(v\), excluding \(\{u,v\}\). Suppose \(\phi(a)=\phi(b)=c\). Then \(a\), \(b\) and \(v\) are three common neighbours of \(u\) and \(c\), so \(a=b\). Hence \(\phi\) is injective. By symmetry \(u\) and \(v\) have the same degree. Since \(G\) is connected and any two adjacent vertices have the same degree, \(G\) is regular.

The SRG parameters \(\lambda\) and \(\mu\) are 2, and by Lemma 2.13 we have
\begin{align*}
&k(k-\lambda-1)=(n-k-1)\mu\\
\implies\quad&n=\frac{k(k-3)}2+1+k.
\end{align*}
Hence \(G\) is strongly regular with parameters
\[(n,k,\lambda,\mu)=\bigg(\,\frac{k(k-3)}2+1+k,\,k,\,2,\,2\,\bigg).\]

\item By the rationality condition the two numbers
\begin{align*}
f,g&=\frac12\bigg(n-1\pm\frac{2k+(n-1)(\lambda-\mu)}{\sqrt{(\lambda-\mu)^2+4(k-\mu)}}\bigg)\\
&=\frac12\bigg(n-1\pm\frac{2k}{\sqrt{k-2}}\bigg)
\end{align*}
are integral. Hence \(\frac{2k}{\sqrt{k-2}}\) must be integral. It follows that \(k-2\) must be a square \(a^2\) so
\[\frac{2k}{\sqrt{k-2}}=\frac{2(a^2+2)}{a}=2a+\frac4a\]
which is integral only if \(a\) is 1, 2 or 4, which implies \(k\) is 3, 6 or 18. \(k=18\) does not satisfy the rationality condition so \(k\) is 3 or 6.

\item The \(k=3\) case is exhibited by \(K_4\). The \(k=6\) case is exhibited by \(K_4\square K_4\), since if if two vertices \(u,v\) are on the same row or column their common neighbours are the other two vertices on that row or column, and otherwise their common neighbours are the two vertices that complete the corners of a rectangle with \(u\) and \(v\).

\end{enumerate}

\item Let \(G\) be a connected graph of order \(n\) such that every vertex has degree at most 3.

\begin{enumerate}

\item Pick a vertex \(u\) in \(G\). Its orbit has size \(n\). Pick a neighbour \(v\) of \(u\). Under the stabiliser of \(u\) it must be mapped to a neighbour of \(u\) so its orbit has size at most 3.

Now pick a new point \(a\) adjacent to some previously chosen point \(b\). If \(a\) is already fixed pick a new point until you pick some \(a\) that is not fixed adjacent to some previously chosen point \(b\). The point \(b\) has a fixed neighbour so it must have degree \(3\), since if it had degree \(2\) then \(a\) would be fixed as the unique other neighbour of \(b\). Then \(a\) must be sent to one of the two other neighbours of \(b\). This determines where the third neighbour of \(b\) gets sent as well. Since \(G\) is connected if we continue in this fashion we will eventually have fixed every point in \(G\). Each step of the process fixes at least two points and contributes a factor of 2 to the size of \(\aut G\). Hence the total contribution after every vertex has been fixed is at most \(2^{\frac{n-2}2}\). The size of \(\aut G\) is therefore at most \(3n2^{\frac{n-2}2}\)

\item \(K_4\) and \(K_{3,3}\) meet the bound since
\begin{gather*}
\abs{\aut K_4}=4!=3(4)2^{\frac{4-2}2}\\
\abs{\aut K_{3,3}}=2!\cdot3!\cdot3!=3(6)2^{\frac{6-2}2}.
\end{gather*}

\item If \(G\) is not vertex-transitive we have \(m<n\) choices for the first vertex in the process above and the bound becomes \(3m2^{\frac{n-2}2}\), so \(G\) must be vertex-transitive.

If \(G\) has a vertex of degree \(m<3\), we can pick it as the first vertex in the process above and the bound becomes \(mn2^{\frac{n-2}2}\), so \(G\) must be 3-regular.

For the bound to be exact each step of the process must fix exactly two points. In particular, in the last step of the process, there must be exactly two points that are not fixed and hence have the same closed neighbourhood if they are adjacent and open neighbourhood if they are not.

Therefore any graph for which the bound is exact must have one of the two graphs below as a subgraph:
\begin{center}
\includegraphics[valign=c, width=3cm]{g}\hspace{3em}\includegraphics[valign=c, width=3cm]{k}
\end{center}
In the first graph the middle vertices belong to at least two 3-cycles, so by transitivity every vertex must belong to at least two 3-cycles. The only way to make the left and right vertices belong to at least two 3-cycles is to join them with an edge, which yields \(K_4\).

In the second graph the top vertex is adjacent to two vertices that have the same open neighbourhood, so by transitivity the left vertex must be adjacent to two vertices that have the same open neighbourhood. By symmetry pick the top and middle vertices. Their third common neighbour cannot be the bottom vertex since the graph must be 3-regular. Therefore we have the following subgraph:
\begin{center}
\includegraphics[valign=c, width=3cm]{kl}
\end{center}
The left vertex is in at least four 4-cycles, so by vertex transitivity the bottom vertex must be in at least four 4-cycles, which is only possible if it connects to the top-right vertex, yielding \(K_{3,3}\).

\end{enumerate}

\end{enumerate}

\end{document}